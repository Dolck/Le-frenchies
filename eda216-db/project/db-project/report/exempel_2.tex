%%%%%%%%%%%%%%%%%%%%%%%%%%%%%%%%%%%%%%%%%%%%%%%%%%%%%%%%%%%%%%%%%%%%%%%%%%%%%%
% Detta �r ett exempel p� ett latexdokument.
% 
% Alla dokument best�r av f�ljande delar:
%
%          \documentclass[optioner]{dokumentklass}
%            ...inst�llningar...
%          \begin{document}
%            ...text...
%          \end{document}
%
% Som ni kanske redan har f�rst�tt �r anv�nds procent (%) f�r
% kommentarer.
%%%%%%%%%%%%%%%%%%%%%%%%%%%%%%%%%%%%%%%%%%%%%%%%%%%%%%%%%%%%%%%%%%%%%%%%%%%%%%

\documentclass[a4paper]{article}

\usepackage[T1]{fontenc}                % F�r svenska bokst�ver
\usepackage[swedish]{babel}             % F�r svensk avstavning och svenska
                                        % rubriker (t ex "inneh�llsf�rteckning)
\title{Programming Project, Database Technology}
\author{Tim Dolck dat11tdo@student.lu.se \\ 
Julian Kron� dat11jkr@student.lu.se \\
Christopher Nilsson dat11cni@student.lu.se}
%\date{}           % Blir dagens datum om det utel�mnas

\begin{document}

\maketitle                      % Skriver ut rubriken som vi
                                % deklarerade ovan med \title, \author
                                % och eventuellt \date
\newpage
\section{Introduction}          % Detta kommando g�r en rubrik

\section{Requirements}

\section{Outline}

\section{Model}

\section{Statements}

\section{Manual}

\end{document}                 % The input file ends with this command.
